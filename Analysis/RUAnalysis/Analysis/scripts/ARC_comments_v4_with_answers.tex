%\documentclass[12pt]{article}
\documentclass[paper=a4, fontsize=11pt]{scrartcl}
\renewcommand{\labelitemi}{$-$}
\renewcommand{\labelitemii}{$-$}
\title{ARC Comments v4}
\usepackage{color}
\usepackage{graphicx}
\usepackage[usenames,dvipsnames]{xcolor}
\author{EXO-12-049}
\date{\today}



\begin{document}
\maketitle
\section{GENERAL}
\begin{itemize}
\item CMS convention is to use lower case in paper and
 section titles for all but the first word:
 Title: "Search for light- and heavy-flavor..."
 
\textcolor{ForestGreen}{Fixed.}
\item CMS convention is to write out "Section" and not to
 abbreviate as "Sec."
 
\textcolor{ForestGreen}{Fixed throughout.}
\item The fact that the limits are based on 100\% branching ratio to qqq for
   light flavor gluinos, and also 100\% for heavy flavor to qqq, where at least 
  one q is a b, is buried on line 228. It should be in the Conclusions 
  where the limits are quoted.
  
\textcolor{ForestGreen}{Added to Conclusion, lines 243-244}
\end{itemize}

\section{SPECIFIC}
\subsection{Abstract}
\begin{itemize}
\item line 4 "model independent" [no hyphen]

\textcolor{ForestGreen}{Fixed.}
\item line 4 "...with a large value of the scalar sum of jet transverse momenta;"

\textcolor{ForestGreen}{Fixed.}
\item line 5 "R-parity-violating" [add hyphen before "violating"]

\textcolor{ForestGreen}{Hyphen added here and throughout.
We had previously removed this hyphen at Kostas's request.}
\end{itemize}

\subsection{Text}
\begin{itemize}

\item 8 "which" $\rightarrow$  "that" [mandatory for a restrictive clause]

\textcolor{ForestGreen}{Fixed.}

\item 22 "...help of bottom-quark jet (b-jet) identification..."

\textcolor{ForestGreen}{Fixed.}

\item 23 "...SM top quark-antiquark (ttbar) events..."

\textcolor{ForestGreen}{Fixed.}

\item 25 "Trigger and object reconstruction"
 [better than "triggering"]
 
\textcolor{ForestGreen}{Done.}

\item 30 "...fourth jet is 60 GeV and for the sixth jet 20 GeV."
 [eliminate hyphens and second "is"]
 
\textcolor{ForestGreen}{Done.}

\item 30 "was" $\rightarrow$  "is" [use consistent tense]

\textcolor{ForestGreen}{Fixed.}

\item 31 eliminate unnecessary and redundant
 "for the entire 2012 running period"
 
\textcolor{ForestGreen}{Fixed.}

\item 31 "Sec." $\rightarrow$  "Section"

\textcolor{ForestGreen}{Fixed.}

\item 33 lower case: "particle-flow algorithm"

\textcolor{ForestGreen}{Done.}

\item 35 "...from calibrated measurements [plural] in the electromagnetic
  calorimeter (ECAL)."
  
\textcolor{ForestGreen}{Fixed.}

\item 40 "...ECAL and hadron calorimeter (HCAL) energies [plural]..."

\textcolor{ForestGreen}{Fixed.}

\item 41 "for" $\rightarrow$  "of"

\textcolor{ForestGreen}{Fixed.}

\item 44 change "y" to "$\eta$"

\textcolor{ForestGreen}{Fixed.}

\item 44 "...in y-$\phi$ space, where y is the rapidity and $\phi$
 the azimuthal angle in the plane perpendicular to the beam axis."
 
\textcolor{ForestGreen}{Added, but using eta instead of y, as requested by your previous comment.}

\item 52 lower case except first word "Generation of simulated events"

\textcolor{ForestGreen}{Fixed.}

\item 56 "like" $\rightarrow$  "as"

\textcolor{ForestGreen}{Fixed.}

\item 59 "...from 200 to 500..."

\textcolor{ForestGreen}{Fixed.}

\item 65 "Event selection"

\textcolor{ForestGreen}{Fixed.}

\item 66 remove unnecessary "offline"

\textcolor{ForestGreen}{Fixed.}

\item 70 remove "using"

\textcolor{ForestGreen}{Done, with rephrasing.}

\item Eq.(1): add comma at the end

\textcolor{ForestGreen}{Done.}

\item Fig.1 caption, line 1: remove "is shown"

\textcolor{ForestGreen}{Done.}

\item 87 "for the diagonal offset DELTA of 110 GeV"

\textcolor{ForestGreen}{Done.}

\item 88: The sentence seems to come out of the blue, without it being clear why
the 6th jet is being used.

\textcolor{ForestGreen}{Added a lead-in sentence about using cut to reduce background (line 89).}

\item 92 "Sec." $\rightarrow$  "Section"

\textcolor{ForestGreen}{Fixed.}

\item Eq.(2): add full stop at the end

\textcolor{ForestGreen}{Added a period.}

\item 99 "We find THAT a sixth...preserves HIGH..."

\textcolor{ForestGreen}{Done.}

\item 102 lower case "combined-secondary-vertex algorithm"
  and add hyphen before "secondary"
  
\textcolor{ForestGreen}{Done.}

\item 104 remove "only"

\textcolor{ForestGreen}{Fixed (line 108).}

\item 104-107 We think the sentence "Its medium... this analysis" should come
after the sentence "We study.... best choice".

\textcolor{ForestGreen}{Done (lines 106-109).}

\item Fig.2 caption:
\item  line 1: "light-flavor decay" [hyphenate]

\textcolor{ForestGreen}{Fixed.}
\item  line 3: "...shows the results for signal events only..."

\textcolor{ForestGreen}{Added, but with the word "distribution" instead of "results".}

\item 108 "...at least one tagged b jet (b tag) increases..."

\textcolor{ForestGreen}{Fixed (line 110).}

\item 109 remove unnecessary "must"

\textcolor{ForestGreen}{Fixed (line 111).}

\item 117 "...which typically contain back-to-back jets and thus have
 a more linear shape."
 
\textcolor{ForestGreen}{Done (line 119).}

\item 120 "...event:" [add colon]

\textcolor{blue}{Comment below was implemented, so equation moved.}

\item Eq.(3): This would be better after the word 'tensor' in line 118, and then
"$S^{\alpha \beta}$ (Eq. 3)"  can be omitted.

\textcolor{ForestGreen}{Done.}

\item 124 hyphenate "signal-plus-background events."

\textcolor{ForestGreen}{Done (line 126).}

\item 127 "Signal and background estimation"

\textcolor{ForestGreen}{Done.}

\item Fig.3 caption, line 2:
 "...gluino models."
 "...normalized to unit area."
 
\textcolor{ForestGreen}{Fixed.}

\item 131 "would be" $\rightarrow$  "are"

\textcolor{ForestGreen}{Fixed (line 133).}

\item 132 "...because the normalization of the background component
 is determined in the fit..."
 
\textcolor{ForestGreen}{Rephrased along the lines of your suggestion (lines 134-135).}

\item 138 "...\\ttbar EVENTS are..."

\textcolor{ForestGreen}{Fixed (line 140).}

\item 138 "by" $\rightarrow$  "using"

\textcolor{ForestGreen}{Fixed (line 140).}

\item 139 "...determined FROM next-to-leading..."
 [eliminate redundant "theoretical"]
 
\textcolor{ForestGreen}{Fixed (lines 141-142).}

\item 140 "The shape of the QCD multijet contribution is taken from..."

\textcolor{ForestGreen}{Fixed.}

\item 154 "...template SHAPE that..."

\textcolor{ForestGreen}{Fixed (line 156).}

\item 164 "acceptance-times-efficiency" [hyphenate jargon expressions]

\textcolor{ForestGreen}{Changed throughout.}

\item 170-171
Is it really true that the acceptance*efficiency flattens because of the
trigger efficiency. I thought that
this was always above 97\%, and so would have only a marginal effect on the
A*e plot.

\textcolor{ForestGreen}{No, the flattening is not related to the trigger efficiency, but rather
the 6th-jet pT cut. The wording has been clarified (line 173).}

\item Fig.4 caption: "acceptance-times-efficiency" [hyphenate]

\textcolor{ForestGreen}{Done.}

\item 175 "...difference with respect to the nominal values..."

\textcolor{ForestGreen}{Used "from" to avoid repeating "with respect to" construction that is
used again in the paragraph (line 177).}

\item 179 "acceptance-times-efficiency" [hyphenate]

\textcolor{ForestGreen}{Done.}

\item 181 "Analogously, a systematic uncertainty is assigned to account
 for the effects of multiple pp collisions in an event (pile-up)
 by reweighing all MC signal samples such that the distribution..."
 [eliminate "(PU)" since it is never used]
 
\textcolor{ForestGreen}{Done (lines 183-185).}

\item 185 "values" $\rightarrow$  "result"

\textcolor{ForestGreen}{Done (line 187).}

\item 191 "acceptance-times-efficiency" [hyphenate]

\textcolor{ForestGreen}{Done.}

\item 191 "...difference between the calculated and parametrized values
  at each..."
  
\textcolor{ForestGreen}{Done (line 193).}

\item 196 "Sec." $\rightarrow$  "Section"

\textcolor{ForestGreen}{Done.}

\item Table.1:
Can it be made clearer whether the numbers in Table 1 refer to 'input'
(i.e. the uncertainty on the listed
quantity) or on the 'output' (i.e. the limit)?

\textcolor{ForestGreen}{Added phrase "included in limit setting" to table header.}

\item Table 1 caption:
 "Signal acceptance-times-efficiency systematic uncertainties."
 [hyphenate as shown, and lower case except for "Signal"]
 
\textcolor{ForestGreen}{Done.}

\item Sect 6.1:
It is implied but not actually stated that the limit setting procedure is
hybrid i.e. frequentist for the main
determination of the limit, but Bayesian for the systematics.

\textcolor{ForestGreen}{Our limits are calculated with an LHC-style modified frequentist method with a frequentist treatment of systematic uncertainties. It employs the "asymptotic" approximation as implemented in RooStats with CLs as the figure of merit. This method has been used in CMS and ATLAS Higgs analyses.}

\item 203 The statement that we set a limit BECAUSE we see no signal gives rise
to what Feldman and Cousins call
'flip-flop' and should be avoided.

\textcolor{ForestGreen}{Rephrased to avoid flip-flopping (line 204).}

\item 208-209 Have any checks been performed to see whether the asymptotic
approximation is OK?

\textcolor{ForestGreen}{We did check with the full CLs calculator,
and this is now mentioned on line 210.}

\item 209 "...is fit using a binned maximum likelihood function with..."

\textcolor{ForestGreen}{Done (line 211), but with hyphen required by PubComm guidelines
for "maximum-likelihood".}

\item 217 "...uncertainty to account for jet resolution..."

\textcolor{ForestGreen}{Done.}

\item 218 "...are included to account for ambiguities in the manner in
 which the MadGraph and Pythia event generators are interfaced."
 [don't repeat MadGraph reference "[24]" from line 138 or that
 for Pythia "[19]" from line 54]
 
\textcolor{ForestGreen}{Done.}

\item 222 "...the SOLID red lines show the NLO plus
 next-TO-leading-logarithm (NLL)..."
 [note missing "to"]
 
\textcolor{ForestGreen}{Fixed.}

\item 223 "...production, and the dashed red lines the corresponding
 one-standard-deviation ($\sigma$) uncertainties, which range..."
 
\textcolor{ForestGreen}{Done.}

\item 225 "To quote numerical results, we conservatively use the
 points where the curve showing the NLO+NLL cross section
 minus one standard deviation uncertainty crosses the
 expected- and observed-limit curves.  In addition,
 we quote the result where the central theory curve
 intersects the limit curves."
 
\textcolor{ForestGreen}{Rephrased, but with a slight variation. It is hard to phrase this concept
 in an elegant way (lines 226-229).}

\item 229 Present results of current document first:
 "The production of RPV gluinos decaying into light-flavor jets is
 excluded at 95\% CL for gluino masses below 650GeV.  This result
 extends our limit of 460GeV [9] obtained with the 2011 CMS data set.
 Gluinos that decay into heavy-flavor jets are excluded for masses
 between ..."
 
\textcolor{ForestGreen}{Done (lines 230-233).}

\item Fig.5 caption:
 "Comparison of data with the background estimate for the inclusive
  analysis.  The background estimate (red solid curve) is obtained
  from a maximum likelihood [no hyphen] fit to the data."
 
\textcolor{ForestGreen}{Done, except the PubComm guidelines specifically require a
 hyphen for "maximum-likelihood fit".}

\item Fig.6 caption:
 line 1: "Comparison of data with the background estimate for
   the heavy-flavor analysis." [to match Fig.5 caption]
 line 2: don't repeat so much analysis detail from the text:
   "The left plot shows the results from the low-mass selection.
   The background contribution from the b-tag control region is
   shown in green [not "blue"] while that from simulated \\ttbar
   events is shown in red."
   
\textcolor{ForestGreen}{Done.}

\item Fig.7 caption:
 line 2: "...the asymptotic calculator, for (left) light-flavor,
   and (right) heavy-flavor gluinos."
 line 3: "...searches, ONE for... and ONE for
 line 5: "The solid red lines SHOW the NLO+NLL predictions
   and the dashed red lines the corresponding one-standard-deviation
   uncertainty bands."
   
\textcolor{ForestGreen}{Done, except we use $\sigma$ that we defined previously.}

\item 239 "model independent" [no hyphen]

\textcolor{ForestGreen}{Fixed.}

\item 240 write out acronyms again for the summary, and you never use "SUSY"
 anywhere else in the paper, so get rid of it here:
 "...using the R-parity-violating (RPV) supersymmetric model..."
 
\textcolor{ForestGreen}{Done.}

\item 241 "...scenarios FOR this..."

\textcolor{ForestGreen}{Fixed.}

\item 242 "...light-flavor jets, [comma] and..."

\textcolor{ForestGreen}{Fixed.}

\item 242 "...at least one bottom-quark jet..."

\textcolor{ForestGreen}{Done (line 243).}

\item 243 "...standard model QCD multijet estimates."

\textcolor{ForestGreen}{Done (line 245).}

\item 244 "...analyzed for THE presence..."

\textcolor{ForestGreen}{Fixed (line 246).}

\item 245 "...the expected standard model background predictions and
 the numbers of selected events."
 
\textcolor{ForestGreen}{Done, except we use "background estimates" and not "predictions"
because, strictly speaking, we do not predict the background but rather estimate it from data
(line 247).}

\item 247 "...at 95\%" [eliminate "a"]

\textcolor{ForestGreen}{Done throughout.}

\item 247 "Heavy-flavor gluinos have been excluded at 95\%..."
 [eliminate "also" and "a"]
 
\textcolor{ForestGreen}{Done (line 249).}
\end{itemize}
\subsection{References}
\begin{itemize}
\item [-] [8] eliminate stutter: "Collaboration Collaboration"

\textcolor{ForestGreen}{Fixed throughout.}

\item  [-] [8] add spaces between "Phys.", "Rev.", and "Lett."
 [this is done correctly in Ref.[22]]
 
\textcolor{ForestGreen}{Fixed.}

\item [-]  [9] Phys.ReV.  $\rightarrow$Phys. Lett.
 Add a space between "Phys." and "Lett."
 The "B" should not be in boldface
 
\textcolor{ForestGreen}{Fixed.}

\item [-]  [10] eliminate stutter: "Collaboration Collaboration"

\textcolor{ForestGreen}{Fixed.}

\item [-]  [10] "1212"  $\rightarrow$  "12" [this is CMS convention for JHEP;
 it is done correctly in Refs.[19] and [24]]
 
\textcolor{ForestGreen}{Fixed.}

\item [-]  [15] add spaces between "Eur.", "Phys.", and "J".
 Also, the "C" should not be in boldface.
 
\textcolor{ForestGreen}{Fixed.}

\item [-]  [16] add space between "Phys." and "Lett."
 The "B" should not be in boldface.

\textcolor{ForestGreen}{Fixed.}

\item  [-] [17] (November, 2011)  $\rightarrow$ (2011)

\textcolor{ForestGreen}{Fixed.}

\item [-]  [20] Nucl. Instr. and Methods A  $\rightarrow$ Nucl. Instrum. Meth. A
 [this is done correctly in Ref.[26]]
 
\textcolor{ForestGreen}{Fixed.}

\item  [-] [23] eliminate stutter: "Collaboration Collaboration"

\textcolor{ForestGreen}{Fixed.}

\item [-]  [28] The European Physical Journal C  $\rightarrow$ Eur. Phys. J. C

\textcolor{ForestGreen}{Fixed.}

\item  [-] [29] (ACAT2010). $\rightarrow$ (ACAT2010),

\textcolor{blue}{This entry comes verbatim from the TDR bib file in SVN.
It is an INPROCEEDINGS, and the format with the period here is created by the tdr command.
The format of this reference exactly matches that of the same reference in a CMS Higgs paper 
(http://arxiv.org/abs/1302.2892). We believe this format is the CMS standard format.}


\end{itemize}
\end{document}

