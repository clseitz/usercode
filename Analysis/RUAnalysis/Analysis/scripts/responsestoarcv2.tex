\documentclass[paper=a4, fontsize=11pt]{scrartcl}
\renewcommand{\labelitemi}{$-$}
\renewcommand{\labelitemii}{$-$}
\title{ARC Comments v1}
\usepackage{color}
\usepackage{graphicx}
\usepackage[usenames,dvipsnames]{xcolor}
\usepackage{hyperref}
% \usepackage[T1]{fontenc}
\author{EXO-12-049}
\date{\today}

\begin{document}
================== GENERAL STATEMENT ======================================== 


We have been satisfied by the answers given to our first set of comments, and 
there are no remaining Type B (physics) questions. Still, some work is needed 
to improve the clarity of the PAS. 


The comments below are based on V5 of the PAS and V8 of the AN note. 


================== PAS TEXT SPECIFIC COMMENTS =============================== 

\begin{itemize}
\item Abstract L4: 
large sum jet pT ==\textgreater large scalar sum of jet transverse momentum 

\textcolor{ForestGreen}{Fixed.}\\

\item Abstract L5: 
R-parity-violating ==\textgreater do we need hyphen between parity and violating? We 
guess not. 

\textcolor{ForestGreen}{Fixed. (RPV seems like a unit, but usage in HEP writing appears to favor only one hyphen.)}\\

\item Abstract L8: 
b jet ==\textgreater b-jet 

\textcolor{ForestGreen}{We have revised this sentence to avoid the use of b~jet
in the abstract.}\\

\item Abstract L9: 
ttbar ==\textgreater italic (line L9) 

\textcolor{ForestGreen}{Fixed.}\\

\item L6: 
LHC and L15 Large Hadron Collider (LHC) 
==\textgreater L6 Large Hadron Collider (LHC) and L15 LHC 

\textcolor{ForestGreen}{Fixed. Thanks for spotting this!}\\

\item L8: 
Change 'limits ON gluinos' to 'limits WHICH EXCLUDE gluinos' 
[Limits are set at all masses] 

\textcolor{ForestGreen}{Fixed.}\\

\item L9/10: 
"An additional search performed as a counting experiment with pp collisions 
at 7 TeV has excluded gluiness below 666 GeV [10]" 

\textcolor{ForestGreen}{Fixed.}\\

\item L14: 
Is reference [11] a proper reference for the 2012 luminosity? 

\textcolor{ForestGreen}{Thank you, we've updated the reference here.}\\

\item L11: 
three-jet hadronic ==\textgreater remove “hadronic” 

\textcolor{ForestGreen}{Fixed.}\\

\item L12: 
pp ==\textgreater Italic 

\textcolor{ForestGreen}{We have switched the text to use the standard CMS shortcut (PpPp)
that will automatically select the appropriate choice for 'pp' following the PubComm
suggestions. For the PAS, it will show up as Roman font, but for the destined journal,
the font will automatically be updated to reflect the journal's requirements.}\\

\item L21/22: 
The terminology 'hadronic resonance, the top quark' is rather unfortunate. 
[Top quark is a quark, not a resonance]. Perhaps you can rephrase? 

\textcolor{ForestGreen}{Sentence rephrased.}\\

\item Paragraph starting at line 102: 
a) Does this apply just to signal combinatorics, and no QCD background? But 
lines 108/9 mention the data as a source of estimate for the bgd, and line 
117 mentions QCD background. So is it that in order to optimise the 
significance, the signal is estimated as the Gaussian part of MC for signal 
(including combinatorics) while the bgd (combinatorics + QCD) is from the 
data?

\textcolor{ForestGreen}{The signal is defined as the Gaussian signal peak only.
The background estimate used in the signal significance calculation is
derived only from data. A few sentences in the paragraph were rephrased
to try to make this point more clearly.}\\

\item b) Include eqn 3 immediately after line 108, rather than inelegantly floating 

\textcolor{ForestGreen}{Fixed.}\\


\item after line 162. 
c) Line 114: 'shown in the TABLE ON THE RIGHT OF figure 2.' 
d) Line 116: '.... a sixth-jet P-T CUT of 110 GeV 
(and similarly on lines 128 and 131) 

\textcolor{ForestGreen}{All fixed.}\\


\item L62: 
"in y-phi space" (the anti-kT clustering uses y, not eta !!!) 

\textcolor{ForestGreen}{Fixed.}\\


\item L62: 
Jet-energy scale ==\textgreater Jet energy scale 

\textcolor{ForestGreen}{Fixed.}\\


\item L63: 
reference [15] is obsolete: replace with the JINST paper on JES/JER 

\textcolor{ForestGreen}{Replaced old reference with JINST paper.}\\


\item L66: 
"… range from 1-5\%..." 

\textcolor{ForestGreen}{Fixed.}\\


\item L67: 
jet’s pseudorapidity and energy ==\textgreater pseudorapidity and energy of jets 
Jet-quality ==\textgreater Jet quality 

\textcolor{ForestGreen}{Both fixed.}\\


\item L74: 
previous analyses ==\textgreater add ref here 

\textcolor{ForestGreen}{References added.}\\


\item L76: 
b jet ==\textgreater b-jet 

\textcolor{ForestGreen}{We defer to the PubComm guidelines here and refrain from the use of the hyphen
in this case (q.v., \url{https://twiki.cern.ch/twiki/bin/viewauth/CMS/Internal/PubGuidelines\#Hyphens})}\\


\item L81: 
Geant4 ==\textgreater GEANT4 

\textcolor{ForestGreen}{Fixed.}\\


\item L88: 
well-motivated ==\textgreater well motivated 

\textcolor{ForestGreen}{Fixed.}\\


\item L99: 
Possible to use the same format of formulae as Eqn (1)? 

\textcolor{ForestGreen}{Fixed.}\\


\item L100: 
vs. ==\textgreater vs. triplet scalar pT (or remove “The triplet invariant mass since 
this already defined in L99) 
plot for a ==\textgreater remove “plot” 

\textcolor{ForestGreen}{Rephrased to use words rather than symbols that seemed a bit awkward.}\\


\item L101: 
of 400 GeV mass ==\textgreater of a mass of 400 GeV 

\textcolor{ForestGreen}{Fixed.}\\


Fig. 1. ==\textgreater Figure 1. 

\textcolor{ForestGreen}{We defer to the PubComm guidelines and keep the use of 'Fig.' here
(q.v., \url{https://twiki.cern.ch/twiki/bin/viewauth/CMS/Internal/PubGuidelines\#Miscellaneous}).}\\


\item L105: 
sixth-jet-pT ==\textgreater sixth-jet pT 

\textcolor{ForestGreen}{Fixed.}\\

\item L107: 
in Fig. 2. ==\textgreater in the left plot in Figure 2. 

\textcolor{ForestGreen}{We defer to the PubComm guidelines and keep the use of 'Fig.' here
(q.v., \url{https://twiki.cern.ch/twiki/bin/viewauth/CMS/Internal/PubGuidelines\#Miscellaneous}).}\\


\item L111: 
the term "standard deviations" is out of context. Better to rephrase using 
the term "resolution". 

\textcolor{ForestGreen}{We have rephrased this accordingly.}\\

\item paragraph from L118-144: 
better split it in two, where the second discusses the sphericity (starting 
from L132). 

\textcolor{ForestGreen}{Fixed.}\\

\item L124, L127, L153: 
b tag ==\textgreater b-tag 

\textcolor{ForestGreen}{We defer to the PubComm guidelines here and refrain from the use of the hyphen
in this case (q.v., \url{https://twiki.cern.ch/twiki/bin/viewauth/CMS/Internal/PubGuidelines\#Hyphens})}\\


\item L125: 'that each ACCEPTED triplet is....' 

\textcolor{ForestGreen}{Rephrased for more clarity.}\\

\item Sphericity: 
The text above eqn 2 appears to apply to b-tagged events. But the captions 
in fig 6 say that it is used in the light gluino search. 

\textcolor{ForestGreen}{The sphericity cut is used for both light- and heavy-flavor searches.
The text has been revised to make this more clear.}\\

\item L131: 
A common value ==\textgreater better way to say? 

\textcolor{ForestGreen}{Sentence rephrased.}\\

\item Eqn.2: 
move "i" under summation 

\textcolor{ForestGreen}{Fixed.}\\

\item L140 \& Fig.2 Caption: 
table ==\textgreater any alternate word? 

\textcolor{ForestGreen}{The tables have been deleted as the ARC suggested, so this bit
of text has been deleted.}\\

\item Section 6, first paragraph: 
a) It seems there are separate fits of eqn 3 to the QCD and the combinatorial 
backgrounds. Are these used in the limit setting, or is it another 4 
parameter fit to the data? Or are the parameters for QCD and for 
combinatorics taken as identical. 

\textcolor{ForestGreen}{We have clarified these phrases in the text as they
were confusing. The function from Eqn3 is fit to the data distribution directly,
and this important point is now clearer in the current version of the PAS.}\\

b) We found the description of the anti-btag confusing. This sounds as if it 
is to kill off the t-pairs. But it then says that the t-pair contribution is 
extracted. Is this because some of the b-pairs from t-pair production turn 
out not to be b-tagged? Please clarify in PAS. 

\textcolor{ForestGreen}{The use of a b-tag veto is to acquire a statistically
independent sample from data that models the QCD background well. The ttbar
and QCD contributions in the b-tagged analysis are treated separately, so an
accurate description of both background shapes is required. This section has
been revised to more accurately reflect our approach.}\\

c) Line 156: Is 'this background' t-pairs? And is it free floating (line 156) 
in the same fit as when QCD is free floating (line 159) {Or is the bgd in 
line 156 QCD?} 
{we think that part of the problem is that it is not clear whether t-pairs 
are considered as part of the bgd.} 

\textcolor{ForestGreen}{The phrase 'this background' has been removed as it
was ambiguous (it did refer to the QCD background). Overall, the revisions
to this section should clarify the entire procedure.}\\

\item L161: 
.. 10\% uncertainty the r = 1 value that … ==\textgreater ??? 

\textcolor{ForestGreen}{Rephrased for clarity.}\\

\item Eqn.3: 
P4 ==\textgreater How about "dN/dx" ? 

\textcolor{ForestGreen}{Fixed.}\\

\item L168: 
limits. ==\textgreater limit on what? Suggest to add more info here. 

\textcolor{ForestGreen}{Phrasing was slightly expanded.}\\

\item L180:
Fig. 6. ==\textgreater Figure 6.

\textcolor{ForestGreen}{We defer to the PubComm guidelines and keep the use of 'Fig.' here
(q.v., \url{https://twiki.cern.ch/twiki/bin/viewauth/CMS/Internal/PubGuidelines\#Miscellaneous}).}\\


\item paragraph 7.1: 
This paragraph needs substantial improvement. The discussion on the 
systematics is too short and too vague. Please discuss each and every one 
uncertainty that you consider in the analysis and summarize them in a table 
(which will contain the uncertainty on the source, not its impact on the 
limit). 

\textcolor{ForestGreen}{We have added much more detail concerning the method
we use to define each systematic uncertainty, as well as providing a summary
table.}\\



\item L185:... (which also... scale), ==\textgreater this is not a part of ISR/FSR 
systematics. so, better to separate it from ISR/FSR sentence.

\textcolor{ForestGreen}{We have rephrased this to remove the scale uncertainties from the ISR/FSR description.}\\


\item L184: 

b tagging ==\textgreater b-tagging 

\textcolor{ForestGreen}{We defer to the PubComm guidelines here and refrain from the use of the hyphen
in this case (q.v., \url{https://twiki.cern.ch/twiki/bin/viewauth/CMS/Internal/PubGuidelines\#Hyphens})}\\


\item Line 188: 
Is the JER independent of jet energy? Also the reference [25] is obsolete 
(replace with the JINST paper on JES/JER). 

\textcolor{ForestGreen}{Reference has been updated. JER has some dependence on jet energy,
but we simply choose a conservative value of 10\% as the uncertainty as per the Exotica
recommendation.}\\

\item Section 7.2: 
The description of the limits is very brief: 
a) What data statistic is used? Is it the profile likelihood ratio? 
b) What would the limits have been without systematics? 
c) Mention that (and how?) the theory error is taken into account. 

\textcolor{ForestGreen}{Following your suggestions we have expanded the limit setting
section including information about  the profile likelihood ratio, which is used in
the analysis. Additionally, the use of the theoretical uncertainties has been described
in more detail. We follow the SUSY groups convention of placing the theory uncertainties
as bands around the nominal value and quoting the limit where the -1 sigma theory line
intersects with the expected and observed curves.}\\

\item L195: 
pile-up ==\textgreater PU 

\textcolor{ForestGreen}{Fixed.}\\

\item L206: 
zero-b-tag ==\textgreater zero b-tag 

\textcolor{ForestGreen}{We have changed the way we reference this data sample.}\\


\item L209: 
Fig. 7. ==\textgreater Figure 7. 

\textcolor{ForestGreen}{We defer to the PubComm guidelines and keep the use of 'Fig.' here
(q.v., \url{https://twiki.cern.ch/twiki/bin/viewauth/CMS/Internal/PubGuidelines\#Miscellaneous}).}\\


\item L210: 
2011 CMS analysis ==\textgreater previous CMS analysis using 2011 data [reference] 

\textcolor{ForestGreen}{Fixed.}\\


\item L215: 
pp ==\textgreater Italic 

\textcolor{ForestGreen}{Please see our previous discussion on this (above).}\\

\item L221: 
large sum jet pT ==\textgreater large scalar sum of jet pT 

\textcolor{ForestGreen}{Fixed.}\\

\item L225-226: 
"These limits … decay" ==\textgreater exactly same as L212-213, so better to write it 
differently. 

\textcolor{ForestGreen}{Fixed.}\\


*** References *** 
==\textgreater Please check the reference carefully!! Some references are already 
published!!!!! 

\textcolor{ForestGreen}{We have reviewed our references and made updates accordingly.}\\


================== COMMENTS ON THE FIGURES ================================== 


\item Figure 1: 
a) write "CMS Simulation" (remove preliminary, which is used only for data) 

\textcolor{ForestGreen}{We defer to the Pubcomm guidelines, which tells us to keep the
``Preliminary'' until going to CWR (q.v.,
	\url{https://twiki.cern.ch/twiki/bin/viewauth/CMS/Internal/PubGuidelines\#Figures_and_tables}).}\\

b) in the caption use the same expression for the Sumjjj ptjet as in eq. 1 

\textcolor{ForestGreen}{Fixed.}\\

\item Figures 2 \& 3 (right) 
We really do not need the usefulness of these tables !!! Every analysis uses 
several optimizations, but (almost) none is shown at the publications. There 
is no particular physics message conveyed by these tables for the general 
audience, they are almost unreadable, and even worse, they can arise numerous 
questions. Therefore, we *strongly* suggest removing them from the PAS. 

\textcolor{ForestGreen}{We have deleted these two tables from the PAS as you've suggested.}\\


\item Figure 2: 
(left) write "CMS Simulation" 

\textcolor{ForestGreen}{We defer to the Pubcomm guidelines, which tells us to keep the
``Preliminary'' until going to CWR (q.v.,
	\url{https://twiki.cern.ch/twiki/bin/viewauth/CMS/Internal/PubGuidelines\#Figures_and_tables}).}\\


\item Tables in Figs 2 and 3: (IF you insist on keeping them !!!) 
a) we are not convinced that it is better to present AVERAGED significances. 
Surely it is the actual significance that is relevant for deciding whether 
a cut is optimal. I don't understand how to look at the numbers in the table, 
and to make an optimal selection. 
b) The numbers on the horizontal axis should be at the centres of the bins, 
rather than at the bin edges. 
c) we are also not convinced that S/sqrt(S+B) is the best measure of 
significance. (Compare, for example, Poisson p-value; or probability of 
getting 3 sigma effect;etc.) 

\textcolor{ForestGreen}{We have deleted these two tables from the PAS as you've suggested.}\\


\item Plot in Fig 3: 
a) Any comment on the discrepancies on the left-hand side of the spectrum? 

\textcolor{ForestGreen}{The overall shape between QCD Monte Carlo and the
data is in very good agreement. There is a small shift in this distribution of the
QCD MC compared to the data, but, as this comparison is purely demonstrative
(no efficiencies are taken from the QCD MC here), it does not detract from
the comparison.}\\


b) Point out that the normalisations of the histograms for different gluino 
masses are arbitrarily forced to be the same as the data, rather than 
proportional to the cross-sections. 

\textcolor{ForestGreen}{Fixed.}\\


\item Fig.3 
right plot: legend on the top-right, move to the right-hand side a little 
bit (see Fig.2) 

\textcolor{ForestGreen}{The table has been deleted.}\\


\item Figs 4 and 5: 
a) add in the PAS the fit residuals on all 3 plots (currently shown only at 
the AN note) 

\textcolor{ForestGreen}{Residuals have been added to the current plots for additional clarity.}\\


b) Fig 4 and right part of fig 5 have chi2/ndf. It is better to have them 
separately rather than just a ratio. Left part of fig 5 has neither. 

\textcolor{ForestGreen}{The main reason for the chi2/ndf is simply to give an
indication of goodness of the fit, thus we prefer not to have them listed separately.
The left part of fig 5 is simply for illustration, with a scaling of the QCD background
(b-tag control region) set to a value determined from the actual fit, which
is not shown here. Instead, we have added a residual distribution to provide
a more quantified level of agreement.}\\


c) why is the "chi" uppercase? 

\textcolor{ForestGreen}{It is now lowercase.}\\


d) caption: x axis ==\textgreater x-axis 

\textcolor{ForestGreen}{Fixed.}\\

\item Fig.5 
a) left plot: 
add "Sphericity \textgreater 0.4" 

\textcolor{ForestGreen}{The left plot is for the low-mass search. The sphericity cut is only used for the
high-mass searches.}\\

1 b tag ==\textgreater 1 b-tag 

\textcolor{ForestGreen}{We defer to the PubComm guidelines here and refrain from the use of the hyphen
in this case (q.v., \url{https://twiki.cern.ch/twiki/bin/viewauth/CMS/Internal/PubGuidelines\#Hyphens})}\\


0-b jet ==\textgreater 0 b-tag jet 

\textcolor{ForestGreen}{We have changed the way we reference this data sample.}\\
%%DD:In either case this is an awkward construction and we should avoid it. I've renamed this
%%To be a control region.

b) right plot: 
1 btag ==\textgreater 1 b-tag 

\textcolor{ForestGreen}{We defer to the PubComm guidelines here and refrain from the use of the hyphen
in this case (q.v., \url{https://twiki.cern.ch/twiki/bin/viewauth/CMS/Internal/PubGuidelines\#Hyphens})}\\


Any RPV gluino distribution as shown in Fig.4? 

\textcolor{ForestGreen}{If the ARC feels strongly about having a gluino signal model added to this
figure, it can be put in.}\\

c) caption: 
sixth-jet-pT ==\textgreater sixth-jet pT 

\textcolor{ForestGreen}{Fixed.}\\

zero-b-jet ==\textgreater zero b-tag jet (or zero b-jet) 

\textcolor{ForestGreen}{We have changed the way we reference this data sample.}\\


\item Fig.6 
a) write "CMS Simulation" 

\textcolor{ForestGreen}{We defer to the Pubcomm guidelines, which tells us to keep the
``Preliminary'' until going to CWR (q.v.,
	\url{https://twiki.cern.ch/twiki/bin/viewauth/CMS/Internal/PubGuidelines\#Figures_and_tables}).}\\


b) caption: needs to be improved

\textcolor{ForestGreen}{Thanks, we've made the caption clearer here.}\\


c) comment in the text the behavior of acceptance x efficiency at high masses 

\textcolor{ForestGreen}{We have added a phrase to describe this behavior, as well
as adding the measured points with their statistical uncertainties.}\\


\item Fig.7 
a) Last legends in the box (e.g. RPV XYZ)==\textgreater replace those by reader friendly way. 

\textcolor{ForestGreen}{We have changed these labels to more accurately reflect the model and remove jargon.}\\

b) why do you need the grid lines? 

\textcolor{ForestGreen}{These have been removed.}\\


\end{itemize}
\end{document}
